%! Author = serox
%! Date = 08/12/2024


\chapter{Qu'est ce qu'un matériau composite ?}


\section{Définition}

Un matériau composite est un matériau composé d'au moins deux autres matériaux différents~\cite{polyvia}.
La combinaison de ces matériaux en crée un nouveau qui possède des propriétés bien plus intéressantes que s’ils étaient séparés.
Ils peuvent notamment être plus résistants aux chocs, à la déformation ou à la chaleur selon ce qu’on recherche.
On peut ainsi sélectionner et exploiter les meilleures caractéristiques techniques de chacun.
Un exemple simple est celui du manteau d'hiver : on associe un tissu imperméable à une matière isolante comme la laine, créant ainsi un vêtement à la fois imperméable et chaud.


\section{De quoi se compose un matériau composite ?}
Comme mentionné dans la définition, un matériau composite est composé d'au moins deux matériaux différents qui ont chacun une utilité technique bien précise.

\subsection{La matrice ou résine}
La matrice entoure les renforts, leur donnent la forme désirée, les protègent des agressions extérieures (comme l’érosion) et assure le bon transfert des efforts mécaniques entre les fibres du renfort.


Il existe trois types de matrice :


\begin{itemize}

    \item
    Les matrices polymères (ou organiques):


    Ce sont les matériaux les plus répandus, on retrouve notamment les thermoplastiques et les thermodurcissables comme la résine époxy ou encore le polypropylène. Ils sont utilisés pour des composites légers et résistants, notamment pour faire des matériaux en aéronautique ou dans le domaine automobile.




    \item
    Les matrices métalliques :


    Ce sont des matrices composées de matériaux tels que l’aluminium ou le titane. Ils sont utilisés pour renforcer la résistance mécanique à haute température. Ils sont donc utilisés pour construire des moteurs ou des turbines.




    \item
    Les matrices céramiques:


    Ce sont des matrices constituées de matériaux comme l'oxyde d'aluminium ou le carbure de silicium. Ils résistent à des températures très élevées et à des environnements corrosifs. Ils sont donc utilisés pour des moteurs de fusées ou encore pour faire des boucliers thermiques dans les navettes spatiales.


\end{itemize}

\subsection{Les renforts}


Un renfort, dans un matériau composite, consiste en un élément structurel (fibres, particules, tissus, etc.) qui est conçu pour apporter des propriétés spécifiques au matériau final, comme la résistance, la rigidité ou la légèreté.


Concrètement, il est constitué d'un matériau de haute performance, tel que :


\begin{itemize}

    \item Fibres : longues ou courtes, elles confèrent au composite des propriétés mécaniques précises. Exemple : fibres de carbone pour leur légèreté et leur solidité.

    \item Particules : elles augmentent la résistance à l'usure ou renforcent la matrice dans des domaines spécifiques. Exemple: les particules métalliques qui peuvent servir à augmenter la résistance mécanique et la conductivité thermique

    \item Tissus ou trames tissées : utilisés pour optimiser les propriétés dans plusieurs directions (multiaxialité). Exemple: Le tissu en fibre de verre utilisé dans des composites à matrice polymère pour des applications comme les carénages, les coques de bateaux ou les pales d'éoliennes.

\end{itemize}


\section{L’histoire du matériau composite}

\indent Les matériaux composites ont une longue histoire remontant à plusieurs milliers d’années. \cite{zhendi}

\vspace{10pt}
En effet, les premières traces de matériaux composites ont été retrouvées pendant l’Antiquité chez les Égyptiens qui utilisaient de la boue mélangée avec de la paille pour fabriquer des briques.
La paille agissait comme un renforcement, réduisant les fissures et augmentant la durabilité.
De plus, ils utilisaient des boîtes en carton et des couches de lin ou de papyrus imbibées de plâtre pour fabriquer des masques mortuaires.


Par la suite, c’est en Asie, sur les arcs médiévaux en Asie, comme les arcs composites mongols, qui combinaient différentes couches de bois, de corne et de tendons.
Chaque matériau apportait des propriétés spécifiques, comme la flexibilité et la résistance à la traction.


Grâce à la révolution industrielle, les évolutions en matière de matériaux composites ont fortement avancé.
En effet, en 1849, Joseph Monier, un jardinier français, a inventé le béton renforcé avec des barres d'acier, donnant naissance à l'un des matériaux composites les plus utilisés.
De plus, en 1907, Léo Baekeland invente la première résine synthétique, la Bakélite.
Elle fut la première matière plastique faite de polymères synthétiques, cela a permis d’ouvrir la voie à des composites modernes.

\vspace{10pt}
Le développement rapide des sciences des matériaux et des matériaux modernes pendant le 20ᵉ siècle a permis la création de composites plus complexes.
La fibre de verre ****dans les années 1930 utilisée avec des résines polymères pour produire des matériaux légers et résistants.
Ensuite les fibres de carbone arrivées dans les années 1950 introduites pour des applications exigeant une résistance et une légèreté extrêmes, comme dans l’aéronautique.
Par la suite les composites avancés, combinant des matrices polymères avec des fibres haute performance comme le Kevlar ou la céramique sont arrivées pour faire des matériaux encore plus résistants et efficaces.


Aujourd’hui, les matériaux composites sont présents partout et de tous les niveaux d’exigences possibles.
On peut en retrouver dans l’aéronautique, l’aérospatiale, l’automobile, la construction ou encore dans les sports et loisirs.

