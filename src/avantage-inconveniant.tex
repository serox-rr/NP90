\chapter{Avantages et inconvénients théoriques}


\section{Les avantages}

Les matériaux composites comportent de nombreux atouts qu’ils sont devenus indispensables dans notre quotidien. \cite{mayer2019b}

\subsubsection{Flexibilité de conception}
En effet, les composites peuvent être façonnés en presque toutes les formes, ce qui permet une grande liberté dans la conception de pièces complexes.
Il est ainsi beaucoup plus simple et moins onéreux de construire une pièce sur mesure en matériau composite que de commander une pièce en matériau métallique, par exemple, c’est-à-dire en soudant plusieurs pièces ou en assemblant à l’aide de vis et de boulons.
De plus, le fait de n’avoir qu’une seule pièce augmente sa résistance.

\subsubsection{La durabilité}

Les matériaux composites ont une longue durée de vie grâce à leur résistance à l’usure, ce qui réduit les coûts de maintenance.
Par ailleurs, les matériaux composites ne sont pas soumis à la corrosion ; ainsi, nous pouvons nous en servir dans des conditions humides (à l’extérieur, au contact de l’eau,…) sans risque de voir notre matériau se dégrader.
Au contraire des matériaux plus standards tels que le fer qui a tendance à rouiller, est plus contraignant, car il demande plus d’entretien.
En effet, il faut régulièrement le contrôler et le traiter à l’aide de peinture pour limiter la corrosion.

\subsubsection{Le réemploie}

Les composites thermoplastiques sont des composites très avantageux.
En effet, grâce à eux, on peut assembler plusieurs thermoplastiques sans colle, mais juste en les faisant légèrement fondre pour être thermosoudés entre eux.
De plus, ce sont des composites qui sont recyclables après broyage ou, même probablement bientôt, après dépolymérisation de la résine thermoplastique.
Enfin, comme le matériau composite est un matériau qui subit très peu la corrosion et qu’il a une grande durabilité, on pense tout de suite à réutiliser le matériau tel quel, c’est-à-dire en s’en resservant pour une autre utilisation que celle prévue initialement.


\section{Les inconvénients}

\subsubsection{Coûts élevés et le processus de fabrication complexe}
La production de composites avancés, comme ceux à base de fibres de carbone, est souvent coûteuse, limitant leur utilisation à des applications haut de gamme telles que l’aéronautique ou les voitures à hautes performances.
La fabrication de pièces composites peut être techniquement exigeante et nécessite souvent des équipements spécialisés, ce qui n’est pas à la portée de tout le monde.

\subsubsection{Difficulté de recyclage}
Les composites, surtout ceux à matrice polymère thermodurcissable, sont difficiles à recycler en raison de leur structure chimique stable~\cite{mayer2019a}.
Le recyclage des matériaux composites est encore un défi majeur, mais les progrès technologiques et les innovations offrent des perspectives prometteuses.
À mesure que la demande pour ces matériaux augmente, des solutions plus durables et économiquement viables devront être adoptées pour limiter leur impact environnemental.

En effet, de nos jours, les moyens de recycler sont plutôt restreints. Il est possible d’incinérer les matériaux usagers pour produire de l’énergie (ce qui n’est pas une manière viable de traiter les déchets et de produire de l’énergie).

Il est également possible de réduire les matériaux composites en les broyant pour pouvoir les réutiliser lors de la fabrication de matériaux composites, c’est ce que l’on appelle le recyclage mécanique.
En effet, le recyclage mécanique consiste à broyer le matériau et à récupérer des broyats qui pourront ensuite être intégrés à une matrice polymère.
Dans certains cas, lorsque le matériau issu du recyclage est destiné à une application à haute valeur ajoutée, cette solution se révèle économiquement intéressante.



\subsubsection{Réparabilité et dégradation sous certains environnements}

Le matériau composite est certes très avantageux : il est léger, résistant aux chocs, durable ; il reste néanmoins possible de le casser et il subit aussi une forme d’usure.
Ainsi, il faut parfois le changer, notamment dans des milieux très exigeants tels que l’aéronautique.
Or, le matériau composite a un indice de réparabilité très faible, voire inexistant, c’est-à-dire qu’il faut changer une pièce entière, alors qu’avec un matériau plus standard, il est parfois possible de réparer.


Néanmoins, il est possible de réparer les matériaux en thermoplastique, car, comme dit plus tôt, nous pouvons les thermosouder entre eux.


Par ailleurs, bien qu'ils soient résistants à la corrosion chimique, certains composites peuvent se détériorer sous l'effet des rayons UV, de l'humidité ou des températures extrêmes.
Ainsi, il est donc impossible d’utiliser certains matériaux dans des conditions extrêmes. Par exemple, on ne va pas utiliser un matériau en fibres végétales dans des cas de chaleur extrême.



