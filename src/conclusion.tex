\chapter{Conclusion}



Aujourd'hui, l'industrie aéronautique est l'une des plus importantes consommatrices de matériaux composites dans le monde.
Malgré leur structure simple dans leur conception fondamentale, ils nécessitent des processus de fabrication sophistiqués et coûteux ce qui explique leur prix de production particulièrement élevé.
Grâce à leurs performances physiques et à la possibilité de les concevoir comme on le souhaite, qui permet la réalisation de formes géométriques complexes et aérodynamiques, ils sont privilégiée dans des secteurs de pointe hautement spécialisés notamment l'aéronautique.
Depuis leur intégration dans la construction d'aéronefs initiée dans les années 1970, ils sont de plus en plus privilégiés par rapport aux matériaux traditionnels comme l'aluminium.

\vspace{15pt}
La question de savoir s'ils sont nuisibles pour l'aéronautique ne se pose plus, ils sont clairement bénéfiques pour cette industrie.
Ils représentent non seulement un atout majeur, mais aussi sont devenu indispensable en répondant à de nouvelles problématiques.
Cependant, dans d'autres domaines qui ne sont industrielles, particulièrement environnemental, ont peu questionné leur pertinence et abusive à longs termes.
En effet, comme mentionné précédemment, les matériaux composites présentent des défis significatifs en matière de recyclage, leur structure complexe et leur composition hétérogène rendant leur traitement en fin de vie particulièrement problématique.

\vspace{15pt}
De plus, le fait que les matériaux composites ont permis de réduire grandement la consommation de carburant pourrait nous amener à penser qu'ils ont un impact positif sur la réduction des émissions de gaz à effet de serre. Mais en réalité, il faut prendre en compte les potentiels effets rebond qu'amènent ces apparentes améliorations. En effet, leur démocratisation ont contribué grandement à rendre le transport aérien plus accessible au grand public. Cela a donc permis de réduire les coûts et ont permis une multiplication des vols, ce qui a naturellement amener à des émissions totales plus importantes, même si les émissions individuelles par vol sont moindres.



