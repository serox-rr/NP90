%! Author = serox
%! Date = 07/12/2024


\chapter*{Introduction}



L'industrie aéronautique est en constante recherche de performance, de légèreté et d'efficacité dans tous les domaines. Elle s'appuie sur des technologiques avancées pour repousser les limites de l'innovation. Parmi celles-ci, les matériaux composites se sont imposés comme des éléments essentiels dans la conception aéronautique moderne. Utilisés pour leur résistance exceptionnelle, leur légèreté et leur durabilité, ils permettent de réduire significativement le poids des appareils et, par conséquent, leur consommation de carburant. Cependant, ces matériaux ne sont pas que bénéfiques : leur fabrication, leur recyclabilité et leur comportement en cas de défaillance suscitent des interrogations. Cela soulève une question importante: les matériaux composites dans l’aéronautique représentent-ils un atout indéniable ou une source de défis potentiellement nuisibles pour l’avenir de l’industrie ?










